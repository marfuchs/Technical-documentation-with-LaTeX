\documentclass[12pt,oneside]{report}
\usepackage{blindtext}
\usepackage[english]{babel}
\usepackage{hyperref}
\usepackage{glossaries}
\makeglossaries

\begin{document}
\tableofcontents
\chapter{Introduction}
%Defining a glossary entry
\newglossaryentry{latex}
{ name=latex,
  description={Software system for document preparation. When writing, the writer uses plain text as opposed to the formatted text found in "What You See Is What You Get" word processors like Microsoft Word, LibreOffice Writer and Apple Pages. The writer uses markup tagging conventions to define the general structure of a document (such as article, book, and letter), to stylise text throughout a document (such as bold and italics), and to add citations and cross-references. A TeX distribution such as TeX Live or MiKTeX is used to produce an output file (such as PDF or DVI) suitable for printing or digital distribution. See \href{https://en.wikipedia.org/wiki/LaTeX}{LaTex on Wikipedia}}
}
%Reference to a glossary entry in the content. 
This is an example how to a glossary entry for 
 \gls{latex} can be used.\\
%Glossary a
\printglossaries
\end{document}

